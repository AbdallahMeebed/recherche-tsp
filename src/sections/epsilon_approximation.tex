\documentclass[../main.tex]{subfiles}
\graphicspath{{\subfix{../images/}}}

\begin{document}

\section{Approximation des problèmes NP-complet}
Un approche commune pour résoudre rapidement (dans un temps polynomial) un problème de classe NP est d'obtenir une approximation qui est prouvé d'être proche de la solution, c'est une $\epsilon$-approximation.

\begin{definition}
Un algorithme est $\epsilon$-approximé pour un problème $P_1$ ssi soit (i) $P_1$ est un problème de maximisation et pour tout instance de $P_1$
\[
|(F^* - \hat{F})/F^*| \leq \epsilon, \quad 0 < \epsilon < 1,
\]
ou bien (ii) $P_1$ est un problème de minimisation et pour tout instance de $P_1$
\[
|(F^* - \hat{F})/F^*| \leq \epsilon, \quad \epsilon > 0,
\]
avec $F^*$ la solution optimale (supposé strictement positive) et $\hat{F}$ la solution approximative obtenue.
\end{definition}

Ce document présente la partie du théorème de \cite{Sahni1976} concernant le problème du voyageur de commerce.

\subsection{Approximation du problème du voyageur de commerce}
Dans cette section, le problème du voyageur  de commerce (TSP pour \emph{Traveling salesman problem})et la preuve du théorème de \cite{Sahni1976} sont présentés. Il est important de noter que si P=NP, l'approximation resterait un problème dans la classe NP-complet (les deux classes sont équivalents). Pour la suite il est supposé que P $\neq$ NP, qui impliquera que n'importe quelle algorithme pour résoudre le problème dans un temps polynomial doit produire des mauvaises approximations dans au moins un instance.

\noindent \emph{Problème de voyageur de commerce}: Soit un graphe $G(N, A)$ un graphe complet avec une fonction poids $\omega : A \rightarrow Z$, trouver le cycle Hamiltonien (le cycle qui passe par tout les sommets exactement une fois) le plus optimal selon les critères suivantes:
\begin{enumerate}
\item Minimiser la longueur du cycle Hamiltonien.
\item Minimiser le temps d'arrivé moyen aux sommets. Le temps d'arrivé est mesuré selon le premier sommet et le poids des arêtes étant le temps pour aller d'un sommet à un autre. Soit $i_1, i_2, \cdots , i_n, i_{n+1}=i_1$ un cycle Hamiltonien, alors le temps d'arrivé $Y_k$ au sommet $i_k$ est:
\[
Y_k = \sum _{j = 1} ^{k-1} \omega (i_j, i_{j+1}), \quad 1 < k \leq n+1
\]
Le temps d'arrivé moyen (à minimiser) est alors
\[
\bar{Y} = \frac{1}{n} \sum _{k = 2} ^{n+1} Y_k = \frac{1}{n} \sum _{j = 1} ^{n} (n+1-j)\omega (i_j, i_{j+1})
\]
\item Minimiser la variance des temps d'arrivés, qui est défini comme
\[
\sigma = \frac{1}{n}\sum_{k = 2}^{n+1} (Y_k - \bar{Y})^2
\]
\end{enumerate}

Afin de prouver que n'importe quelle approximation de ce problème est de classe au moins NP-complet, ce problème sera réduit à un problème prouvé être NP-complet selon \cite{Karp1972}: Vérifier si un graphe contient un cycle Hamiltonien.

\begin{theorem}
L'$\epsilon$-approximation du problème de voyageur des commerces est NP-complète.
\end{theorem}

\begin{proof}
Soit $G(N,A)$ un graphe quelconque. Chaque critère d'optimisation sera traité séparément:

1. Trouver un cycle Hamiltonien $\propto \! \epsilon$-approximation du voyageur de commerce (minimiser la longueur du cycle): Soit $G_1(V, E)$ un graphe complet ($E = \{(u,v)\ |\ u \neq v,\ u,v \in V\}$) avec $V = N$ et $n = |N|$. La fonction de poids $\omega : E \rightarrow Z$ est définie:
\[
\omega \{u, v\} = \begin{cases}
\text{1 si } (u,v)\in A,\\
k \text{ sinon}
\end{cases}
\]
$k$ est une valeur à choisir ultérieurement. Pour $k > 1$, la solution du TSP sur $G_1$ aura une longueur $n$ ssi $G$ contient un cycle Hamiltonien, sinon la solution aura une longueur au moins $k + n -1$.

Si on choisi $k \geq (1+\epsilon)n$, il suffit juste d'évaluer la solution approximative: si elle est inférieure ou égale à $(1+ \epsilon)n$ (ne pas oublier l'erreur commise par l'approximation) alors $G$ contient un cycle Hamiltonien. Sinon, $G$ n'en contient pas. Cela revient à résoudre le problème de trouver si un graphe contient un cycle Hamiltonien (un problème NP-complet).

2. Trouver un cycle Hamiltonien $\propto \! \epsilon$-approximation du voyageur de commerce (minimiser le temps d'arrivé moyen): Soit $G_1(V,E)$ comme défini ci-dessus. Le temps d'arrivé moyen de $G_1$ est au maximum $(n+1)/2$ ssi $G$ contient un cycle Hamiltonien.

Sinon, $\bar{Y} \geq k/n + (n-1)/2$. Si on choisi $k > (1+\epsilon)n(n+1)/2$, il suffit juste d'évaluer la solution approximative: si elle est inférieur ou égale à $(1+\epsilon)(n+1)/2$, alors la solution exacte sera $(n+1)/2$ et donc $G$ contient un cycle Hamiltonien. Sinon, $G$ n'en contient pas.

\begin{figure}[ht]
    \centering
    \incfig{epsi_approx}
    \caption{Construction de $G_1$, les lignes pointillées représentent les arêtes supplémentaires, exclusives à $G_1$}
    \label{fig:epsi_approx}
\end{figure}

3. Trouver un cycle Hamiltonien $\propto \! \epsilon$-approximation du voyageur de commerce (minimiser la variance du temps d'arrivé): Utilisant, $G(N,A)$, on construit un graphe $G_1(N_1, A_1)$ (voir Figure \ref{fig:epsi_approx}) avec:
\[
N_1 = N \cup \{\alpha, \beta, \gamma, \delta \}, \ A_1 = A \cup \{(r, \alpha),(\alpha, \beta),(\beta, \gamma),(\gamma, \delta)\} \ \cup  \{(\delta, z) +| (r,z)\in A\}
\]
pour $r$ comme sommet quelconque dans $G$. Il est évident que $G_1$ contient un cycle Hamiltonien ssi $G$ en contient. Le TSP est obtenu avec le graphe $G_2(N_2, A_2)$ avec $N_2 = N_1$ et $A_2 = \{(u,v)\ |\ u \neq v,\ u,v \in V\}$ (graphe complet) avec la fonction de poids $\omega : A_2 \rightarrow Z$ est définie:
\[
\omega \{u, v\} = \begin{cases}
\text{1 si } (u,v)\in A_1,\\
k \text{ sinon}
\end{cases}
\]
Lemme \ref{lem:lemma} obtient une borne inférieur de $\sigma$.
\begin{lemma}
\label{lem:lemma}
Pour $k > \sqrt{\frac{(1+\epsilon)(n)(n-1)(n+1)}{3}}$ et $\epsilon >0$, $G_2$ contient un cycle Hamiltonien de variance $\sigma \leq \frac{(n-1)(n+1)}{12}$ ssi $G_1$ est Hamiltonien. Sinon, $\sigma > \frac{(1+\epsilon)(n-1)(n+1)}{12}$
\end{lemma}
\begin{proof}
Si $G_1$ est Hamiltonien alors ce cycle est aussi dans $G_2$ avec le poids de chaque arêtes = 1. Alors $\bar{Y} = (n+1)/2$ et
\[
\sigma = \frac{1}{n}\sum_{k = 2}^{n+1} (Y_k - \bar{Y})^2 = \frac{1}{n}\sum_{k = 2}^{n+1} Y_k^2 - \bar{Y}^2 = \frac{2n^2+3n+1}{6} - \frac{(n+1)^2}{4} = \frac{(n+1)(n-1)}{12}
\]

Si $G_1$ n'est pas Hamiltonien, alors chaque cycle dans $G_2$ doit contenir un arête de poids $k$. Soit le tour optimal $i_1,i_2,\cdots , i_n,i_{n+1}$ avec $i_1=i_{n+1}=\beta$. Il y a 3 cas:

\emph{Cas 1:} $\omega (\beta, i_2) = 1$ et $\omega (i_j, i_{j+1})=k$ pour $j$, $1 < j \leq n$: Dans ce cas $Y_2=1$ et $Y_{n+1} \geq k+n-1$. Si $\bar{Y} \geq k/2+1$ alors $|Y_2 - \bar{Y}| \geq k/2$. Si $\bar{Y} < k/2+1$ alors $|Y_{n+1} - \bar{Y}| \geq k/2 + 1$. Dans tous les cas pour $k > \sqrt{\frac{(1+\epsilon)(n)(n-1)(n+1)}{3}}$:
\[
\sigma \geq \frac{(k/2)^2}{n} = \frac{k^2}{4n} > \frac{(1+\epsilon)(n-1)(n+1)}{12}
\]

\emph{Cas 2:} $\omega (\beta, i_2) = k$ et tout le reste des arêtes ont un poids 1. Ceci implique que $i_n=\alpha$ ou $i_n=\gamma$ car leurs arêtes incidents sont les seules qui relient à $\beta$ avec un poids 1. Sans perte de généralité on suppose que $i_n=\alpha$. Puisque $\gamma$ est un sommet dans le cycle, il doit y a voir deux sommets distincts incidents $u$ et $v$ et $v \neq \beta \neq u$ car $\beta$ est déjà dans le cycle et $u \neq i_2$ car $(\beta, \gamma)$ n'est pas dans le cycle. Selon la construction de $G_2$ il est évident que $(\beta, \gamma)$ et $(\gamma, \delta)$ sont les seules qui ont un poids 1 est incidents à $\gamma$. Il n'y a que $\delta$ qui disponible alors un cycle avec seulement $\omega (\beta, i_2) = k$ et le reste des arêtes de poids 1 est impossible.

\emph{Cas 3:} $\omega (\beta, i_2) = k$ et $\omega (i_j, i_{j+1})=k$. Dans ce cas $Y_2=k$ et $Y_{n+1} \geq 2k+n-2$. Si $\bar{Y} \geq 3k/2$ alors $|Y_2 - \bar{Y}| \geq k/2$. Si $\bar{Y} < 3k/2$ alors $|Y_{n+1} - \bar{Y}| \geq k/2$. Par la suite $\sigma > \frac{(1+\epsilon)(n+1)(n-1)}{12}$ (voir Cas 1).

Ceci prend en considération tous les cas dans lesquels $G_1$ n'est pas Hamiltonien.
\end{proof}

Il est maintenant facile de voir que pour obtenir une approximation sur la variance revient à une réduction du problème de chercher un cycle Hamiltonien, et alors le problème $\epsilon$-approximatif du voyageur des commerces est NP-complet.

\end{proof}

\subsection{Conclusion}
Pour résumer, tout algorithme $\epsilon$-approximative du TSP est NP-complet. Ceci à été établi en réduisant le problème à un problème prouvé d'être NP-complet. Autrement dit, si un algorithme peut approximer la solution du TSP pour n'importe quelle graphe, il est au moins assez difficile que résoudre le problème: Est-ce que ce graphe contient un cycle Hamiltonien? La solution de cette question n'est pas obtenue dans un temps polynomial et donc soit l'algorithme approximative ne résolve pas le problème dans un temps polynomial ou bien il n'est pas $\epsilon$-approximative.

\subsection{Classification du voyageur de commerce}
Le problème du voyageur de commerce peut être considéré sous plusieurs classe de complexité selon le problème posé. La version d'un problème de décision: est-ce qu'il existe une solution pour un graphe $G(V,A)$ ?revient à demander s'il existe un cycle Hamiltonien, car s'il existe au moins un cycle alors il y aura une solution. Cette question est donc en classe NP-complet (elle peut être réduit à un problème NP-complet). La version de la question avec un seuil: Est-ce qu'il existe une solution avec un coût total inférieur à $x$? Cette formulation place le problème aussi dans la classe NP-complet, il est facile de vérifier le poids d'un cycle Hamiltonien (somme des arêtes) et donc la vérification d'une solution est faite rapidement.

Par contre, la version \og complète \fg{}: Trouver le cycle Hamiltonien le moins lourd, est placé dans NP-difficile. Premièrement, ce problème n'est plus un problème de décision, c'est un problème de fonction. Ensuite, trouver une solution est au moins assez difficile que trouver un cycle Hamiltonien, donc le problème ne peux pas être résolu dans un temps polynomial (supposant que P$\neq$ NP). En plus, il n'existe pas de méthode rapide pour vérifier si un cycle Hamiltonien dans le graphe est en faite le cycle le moins lourd; une solution donnée ne peut pas être vérifiée dans un temps polynomial.


\end{document}